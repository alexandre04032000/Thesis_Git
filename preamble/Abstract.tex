\chapter*{Abstract}

This dissertation addresses the parametric verification of real-time systems through the use of the IMITATOR model checker. The later is based on parametric timed automata, an extension of timed automata where parameters, variables representing unknown timing values, are incorporated into guards and invariants. This approach provides a powerful formalism for modelling, simulating, and formally verifying an real-time systems. Throughout this work, IMITATOR has been applied to different case studies, including ATM Machine, Coffe Machine, etc.. . In addition, the UPPEX tool is considered, which extends UPPAAL models by integrating annotations and configurations defined in Microsoft Excel files. These files contain tables that specify how to adapt the blocks corresponding to annotation commands up to the next empty line. This functionality enables the efficient modelling and verification of multiple system variants. Since UPPAAL is primarily designed for timed automata, the contribution of this dissertation lies in adapting UPPEX, which harbours parametricity, to support execution with IMITATOR as a backend. This integration thus enhances UPPEX by providing access to the richer formalism of parametric timed automata, thereby offering a more complete and expressive framework for the verification of real-time systems.
\paragraph{Keywords} IMITATOR, Parametric Timed Automata, Timed Automata, UPPAAL, Parametric Verification, Model Checker 

\cleardoublepage

\chapter*{Resumo}

%O model checker IMITATOR realiza a verificação paramétrica de sistemas de tempo real, utilizando redes de parametric timed automata. Este tipo de autómato é uma subclasse dos autómatos temporizados, no qual são introduzidos parâmetros — variáveis que representam valores temporais desconhecidos — nas guardas e invariantes. Trata-se de um formalismo poderoso para simular e verificar formalmente sistemas críticos de tempo real. O IMITATOR foi capaz de verificar diversos casos de estudo da literatura e da indústria, como protocolos de comunicação, circuitos assíncronos de hardware e problemas de escalonamento com períodos incertos. Existe também a ferramenta Uppex, que estende os modelos do UPPAAL, integrando anotações e configurações especificadas em ficheiros Microsoft Excel (com o mesmo nome base), contendo tabelas que descrevem como adaptar o bloco correspondente a um comando de anotação, até à próxima linha em branco. Esta funcionalidade facilita a modelação e verificação eficiente de múltiplas variantes de um sistema. Como o UPPAAL trabalha principalmente com autómatos temporizados, o objetivo deste trabalho é adaptar a ferramenta para que suporte também o funcionamento com o IMITATOR como backend, tornando-a mais completa e expressiva, ao permitir o acesso ao novo formalismo dos autómatos temporais paramétricos.

Esta dissertação aborda a verificação paramétrica de sistemas em tempo real através do uso do verificador de modelos IMITATOR. O IMITATOR baseia-se em redes de autómatos temporizados paramétricos, uma extensão dos autómatos temporizados onde os parâmetros, variáveis que representam valores temporais desconhecidos, são incorporados nas guardas e invariantes. Esta abordagem fornece um formalismo poderoso para modelar, simular e verificar formalmente sistemas críticos em tempo real. Ao longo deste trabalho, o IMITATOR foi aplicado a diferentes estudos de caso, incluindo caixas automáticas (ATM), máquinas de café, entre outros. Adicionalmente, considera-se a ferramenta UPPEX, que estende modelos UPPAAL através da integração de anotações e configurações definidas em ficheiros Microsoft Excel. Estes ficheiros contêm tabelas que especificam como adaptar os blocos correspondentes a comandos de anotação até à próxima linha vazia. Esta funcionalidade permite uma modelação e verificação eficiente de múltiplas variantes do sistema. Uma vez que o UPPAAL é concebido principalmente para autómatos temporizados, a contribuição desta dissertação reside na adaptação do UPPEX para suportar a execução com o IMITATOR como backend. Esta integração potencia a ferramenta, proporcionando acesso ao formalismo mais rico dos autómatos temporizados paramétricos, oferecendo assim um quadro mais completo e expressivo para a verificação de sistemas em tempo real.

\paragraph{Palavras-chave} IMITATOR, Autómatos Temporizados Paramétricos, Autómatos Temporizados, UPPAAL, Verficação Paramétrica, Model Checker 

\cleardoublepage
