\chapter*{Abstract}

The IMITATOR model checker performs parametric verification of real-time systems using networks of parametric timed automata. This type of automaton is a subclass of timed automata in which parameters — variables representing unknown timing values — are introduced in guards and invariants. It is a powerful formalism for simulating and formally verifying critical real-time systems. IMITATOR has successfully verified various case studies from both the literature and industry, such as communication protocols, asynchronous hardware circuits, and schedulability problems with uncertain periods. There is also the Uppex tool, which extends UPPAAL models by integrating annotations and configurations specified in Microsoft Excel files (with the same base name), containing tables that describe how to adapt the block corresponding to an annotation command, up to the next empty line. This functionality facilitates the efficient modelling and verification of multiple system variants. As UPPAAL mainly works with timed automata, the goal of this work is to adapt the tool to also support execution with IMITATOR as a backend, making it more complete and expressive by enabling access to the richer formalism of parametric timed automata.

\paragraph{Keywords} IMITATOR, Parametric Timed Automata, Timed Automata, UPPAAL, Verficação Paramétrica, Model Checker 

\cleardoublepage

\chapter*{Resumo}

O model checker IMITATOR realiza a verificação paramétrica de sistemas de tempo real, utilizando redes de parametric timed automata. Este tipo de autómato é uma subclasse dos autómatos temporizados, no qual são introduzidos parâmetros — variáveis que representam valores temporais desconhecidos — nas guardas e invariantes. Trata-se de um formalismo poderoso para simular e verificar formalmente sistemas críticos de tempo real. O IMITATOR foi capaz de verificar diversos casos de estudo da literatura e da indústria, como protocolos de comunicação, circuitos assíncronos de hardware e problemas de escalonamento com períodos incertos. Existe também a ferramenta Uppex, que estende os modelos do UPPAAL, integrando anotações e configurações especificadas em ficheiros Microsoft Excel (com o mesmo nome base), contendo tabelas que descrevem como adaptar o bloco correspondente a um comando de anotação, até à próxima linha em branco. Esta funcionalidade facilita a modelação e verificação eficiente de múltiplas variantes de um sistema. Como o UPPAAL trabalha principalmente com autómatos temporizados, o objetivo deste trabalho é adaptar a ferramenta para que suporte também o funcionamento com o IMITATOR como backend, tornando-a mais completa e expressiva, ao permitir o acesso ao novo formalismo dos autómatos temporais paramétricos.

\paragraph{Palavras-chave} IMITATOR, Parametric Timed Automata, Timed Automata, UPPAAL, Verficação Paramétrica, Model Checker 

\cleardoublepage
